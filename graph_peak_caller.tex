% Template for PLoS
% Version 3.5 March 2018
%
% % % % % % % % % % % % % % % % % % % % % %
%
% -- IMPORTANT NOTE
%
% This template contains comments intended 
% to minimize problems and delays during our production 
% process. Please follow the template instructions
% whenever possible.
%
% % % % % % % % % % % % % % % % % % % % % % % 
%
% Once your paper is accepted for publication, 
% PLEASE REMOVE ALL TRACKED CHANGES in this file 
% and leave only the final text of your manuscript. 
% PLOS recommends the use of latexdiff to track changes during review, as this will help to maintain a clean tex file.
% Visit https://www.ctan.org/pkg/latexdiff?lang=en for info or contact us at latex@plos.org.
%
%
% There are no restrictions on package use within the LaTeX files except that 
% no packages listed in the template may be deleted.
%
% Please do not include colors or graphics in the text.
%
% The manuscript LaTeX source should be contained within a single file (do not use \input, \externaldocument, or similar commands).
%
% % % % % % % % % % % % % % % % % % % % % % %
%
% -- FIGURES AND TABLES
%
% Please include tables/figure captions directly after the paragraph where they are first cited in the text.
%
% DO NOT INCLUDE GRAPHICS IN YOUR MANUSCRIPT
% - Figures should be uploaded separately from your manuscript file. 
% - Figures generated using LaTeX should be extracted and removed from the PDF before submission. 
% - Figures containing multiple panels/subfigures must be combined into one image file before submission.
% For figure citations, please use "Fig" instead of "Figure".
% See http://journals.plos.org/plosone/s/figures for PLOS figure guidelines.
%
% Tables should be cell-based and may not contain:
% - spacing/line breaks within cells to alter layout or alignment
% - do not nest tabular environments (no tabular environments within tabular environments)
% - no graphics or colored text (cell background color/shading OK)
% See http://journals.plos.org/plosone/s/tables for table guidelines.
%
% For tables that exceed the width of the text column, use the adjustwidth environment as illustrated in the example table in text below.
%
% % % % % % % % % % % % % % % % % % % % % % % %
%
% -- EQUATIONS, MATH SYMBOLS, SUBSCRIPTS, AND SUPERSCRIPTS
%
% IMPORTANT
% Below are a few tips to help format your equations and other special characters according to our specifications. For more tips to help reduce the possibility of formatting errors during conversion, please see our LaTeX guidelines at http://journals.plos.org/plosone/s/latex
%
% For inline equations, please be sure to include all portions of an equation in the math environment.  For example, x$^2$ is incorrect; this should be formatted as $x^2$ (or $\mathrm{x}^2$ if the romanized font is desired).
%
% Do not include text that is not math in the math environment. For example, CO2 should be written as CO\textsubscript{2} instead of CO$_2$.
%
% Please add line breaks to long display equations when possible in order to fit size of the column. 
%
% For inline equations, please do not include punctuation (commas, etc) within the math environment unless this is part of the equation.
%
% When adding superscript or subscripts outside of brackets/braces, please group using {}.  For example, change "[U(D,E,\gamma)]^2" to "{[U(D,E,\gamma)]}^2". 
%
% Do not use \cal for caligraphic font.  Instead, use \mathcal{}
%
% % % % % % % % % % % % % % % % % % % % % % % % 
%
% Please contact latex@plos.org with any questions.
%
% % % % % % % % % % % % % % % % % % % % % % % %

\documentclass[10pt,letterpaper]{article}
\usepackage[top=0.85in,left=2.75in,footskip=0.75in]{geometry}

% amsmath and amssymb packages, useful for mathematical formulas and symbols
\usepackage{amsmath,amssymb}
\usepackage{booktabs}
\usepackage{tabularx}
\newcolumntype{b}{>{\hsize=.49\hsize}X}
\newcolumntype{c}{>{\hsize=1.4\hsize}X}

% Use adjustwidth environment to exceed column width (see example table in text)
\usepackage{changepage}
\usepackage{tabularx}

% Use Unicode characters when possible
\usepackage[utf8x]{inputenc}

% textcomp package and marvosym package for additional characters
\usepackage{textcomp,marvosym}

% cite package, to clean up citations in the main text. Do not remove.
\usepackage{cite}

% Use nameref to cite supporting information files (see Supporting Information section for more info)
\usepackage{nameref,hyperref}

% line numbers
\usepackage[right]{lineno}

% ligatures disabled
\usepackage{microtype}
\DisableLigatures[f]{encoding = *, family = * }

% color can be used to apply background shading to table cells only
\usepackage[table]{xcolor}

% array package and thick rules for tables
\usepackage{array}

% create "+" rule type for thick vertical lines
\newcolumntype{+}{!{\vrule width 2pt}}

% create \thickcline for thick horizontal lines of variable length
\newlength\savedwidth
\newcommand\thickcline[1]{%
  \noalign{\global\savedwidth\arrayrulewidth\global\arrayrulewidth 2pt}%
  \cline{#1}%
  \noalign{\vskip\arrayrulewidth}%
  \noalign{\global\arrayrulewidth\savedwidth}%
}

% \thickhline command for thick horizontal lines that span the table
\newcommand\thickhline{\noalign{\global\savedwidth\arrayrulewidth\global\arrayrulewidth 2pt}%
\hline
\noalign{\global\arrayrulewidth\savedwidth}}


% Remove comment for double spacing
%\usepackage{setspace} 
%\doublespacing

% Text layout
\raggedright
\setlength{\parindent}{0.5cm}
\textwidth 5.25in 
\textheight 8.75in

% Bold the 'Figure #' in the caption and separate it from the title/caption with a period
% Captions will be left justified
\usepackage[aboveskip=1pt,labelfont=bf,labelsep=period,justification=raggedright,singlelinecheck=off]{caption}
\renewcommand{\figurename}{Fig}

% Use the PLoS provided BiBTeX style
\bibliographystyle{plos_latex_template.bbl}

% Remove brackets from numbering in List of References
\makeatletter
\renewcommand{\@biblabel}[1]{\quad#1.}
\makeatother



% Header and Footer with logo
\usepackage{lastpage,fancyhdr,graphicx}
\usepackage{epstopdf}
%\pagestyle{myheadings}
\pagestyle{fancy}
\fancyhf{}
%\setlength{\headheight}{27.023pt}
%\lhead{\includegraphics[width=2.0in]{PLOS-submission.eps}}
\rfoot{\thepage/\pageref{LastPage}}
\renewcommand{\headrulewidth}{0pt}
\renewcommand{\footrule}{\hrule height 2pt \vspace{2mm}}
\fancyheadoffset[L]{2.25in}
\fancyfootoffset[L]{2.25in}
\lfoot{\today}

%% Include all macros below

\newcommand{\lorem}{{\bf LOREM}}
\newcommand{\ipsum}{{\bf IPSUM}}

%% END MACROS SECTION


\begin{document}
\vspace*{0.2in}

% Title must be 250 characters or less.
\begin{flushleft}
{\Large
\textbf\newline{Graph Peak Caller: calling ChIP-Seq Peaks on Graph-based Reference Genomes} % Please use "sentence case" for title and headings (capitalize only the first word in a title (or heading), the first word in a subtitle (or subheading), and any proper nouns).
}
\newline
% Insert author names, affiliations and corresponding author email (do not include titles, positions, or degrees).
\\
Ivar Grytten\textsuperscript{1\Yinyang*},
Knut D. Rand\textsuperscript{2\Yinyang},
Ingrid K. Glad\textsuperscript{2},
Geir O. Storvik\textsuperscript{2},
Alexander J. Nederbragt\textsuperscript{1,3},
Geir K. Sandve\textsuperscript{1},
\\
\bigskip
\textbf{1} Department of informatics, University of Oslo, Oslo, Norway
\\
\textbf{2} Department of Mathematics, University of Oslo, Oslo, Norway
\\
\textbf{3} Department of Biosciences, University of Oslo, Oslo, Norway
\\
\bigskip

% Insert additional author notes using the symbols described below. Insert symbol callouts after author names as necessary.
% 
% Remove or comment out the author notes below if they aren't used.
%
% Primary Equal Contribution Note
\Yinyang These authors contributed equally to this work.

% Additional Equal Contribution Note
% Also use this double-dagger symbol for special authorship notes, such as senior authorship.
%\ddag These authors also contributed equally to this work.

% Current address notes
% \textcurrency Current Address: Department of in/Program/Center, Institution Name, City, State, Country % change symbol to "\textcurrency a" if more than one current address note
% \textcurrency b Insert second current address 
% \textcurrency c Insert third current address

% Deceased author note
%\dag Deceased

% Group/Consortium Author Note
%\textpilcrow Membership list can be found in the Acknowledgments section.

% Use the asterisk to denote corresponding authorship and provide email address in note below.
* ivargry@ifi.uio.no

\end{flushleft}
% Please keep the abstract below 300 words
\section*{Abstract}

Graph-based representations are considered to be the future for reference genomes, as they allow integrated representation of the steadily increasing data on individual variation. Currently available tools allow \emph{de novo} assembly of  graph-based reference genomes, alignment of new read sets to the graph representation as well as certain analyses like variant-calling. We here present a first method for calling peaks on read data aligned to a graph-based reference genome. The method is a graph generalization of the peak caller MACS2 and is implemented in an open source tool, \emph{Graph Peak Caller}. By using the existing tool \emph{vg} to build a pan-genome of \emph{Arabidopsis thaliana}, we validate our approach by showing that Graph Peak Caller with a pan-genome reference graph can detect peaks more accurately than MACS2 with a linear reference genome. 


% Please keep the Author Summary between 150 and 200 words
% Use first person. PLOS ONE authors please skip this step. 
% Author Summary not valid for PLOS ONE submissions.   
\section*{Author summary}
We present Graph Peak Caller, the first peak caller for detecting regions associated to transcription factor binding by using \emph{graph-based reference genomes}. Graph-based reference genomes have become popular as an alternative to the well established “linear” reference genomes, as they are able include known variation within the population, which makes mapping genomic reads to the reference more accurate. Detecting regions associated to transcription factor binding (peaks) is important in order to understand how transcription factors might affect genes. Graph Peak Caller utilizes the benefit of graph-based reference genomes by being able to detect peaks from reads mapped to such references. We show that this leads to more accurate detection of peaks as compared to existing methods that use linear reference genomes.

\linenumbers

% Use "Eq" instead of "Equation" for equation citations.
\section*{Introduction}
Transcription factors are known to play a key role in gene regulation, and detecting regions associated with transcription factor binding is an important step in understanding their function. The most common technique used to detect transcription factor binding sites is through \emph{ChIP-seq}, combining chromatin immunoprecipitation (ChIP) assays with sequencing. Obtaining putative binding regions from ChIP-seq data is done using computational techniques known collectively as performing \emph{peak calling}. Several \emph{peak callers}, programs to perform peak calling, have been developed for this purpose, for example MACS2~\cite{macs} and SPP~\cite{spp}. Common for all current peak callers is that they take reads mapped to a \emph{linear} reference genome, such as GRCh38, as input. 

Graph-based reference genomes have been used in a wide variety of genomic analyses~\cite{de_novo_assembly,velvet}. The software package \emph{vg} supports mapping reads to a graph-based reference genome with potentially increased accuracy~\cite{vg, genome_graphs} as compared to mapping reads to a standard linear reference genome using tools like BWA-mem~\cite{bwa_mem} or Bowtie~\cite{bowtie}. Several types of genomic analyses, such as variant calling and haplotyping, can now be performed using graph-based references~\cite{vg, genome_graphs}. However, no tool currently exists for performing peak calling on graph-based reference genomes. 

% Results and Discussion can be combined.
\section*{Results}
We present Graph Peak Caller, a first method for detecting transcription factor binding events from ChIP-seq reads mapped to a graph-based reference genome. Graph Peak Caller is based on the same principles used by MACS2 (see Fig~\ref{fig1} for an overview), and  is able to call peaks with or without a set of control alignments. As input, it supports alignments in the Graph Alignment\/Map format (GAM) from vg, as well as reads represented as genomic intervals using the Offset Based Graph Python package~\cite{rand} Graph Peak Caller can be run from the command line, and is also available through Galaxy at \url{https://hyperbrowser.uio.no/graph-peak-caller}. In the Github repository at \url{http://github.com/uio-bmi/graph_peak_caller}, we provide a simple tutorial on how to use vg and Graph Peak Caller together to go from raw ChIP-seq reads to peaks. 


% Place figure captions after the first paragraph in which they are cited.
\begin{figure}[!h]
\caption{{\bf Overview of how Graph Peak Caller works.} Example of peak calling on an example graph (nodes in gray, edges in black). After raw reads (a) in the form of input reads (blue) and control reads (red) are mapped to the graph-based reference genome (b), the fragment pileup is created (c) by extending the forward input alignments (shown as resulting blue fragments) and reverse input alignments (yellow fragments) along all possible paths in their corresponding direction. A background track is created by projecting the alignments resulting from the control reads onto a linear path and calculating a local average  read counts (Here shown with a small window size). Then the linear track is projected back to the graph again (not shown). The fragment pileup counts are treated as counts and the background track as rates in a poisson-distribution, and p-values are computed for each position for the observed \emph{count}, given the corresponding rate. Adjusted q-values are computed to control the false discovery rate (d). A set of peak candidates are selected by thresholding the q-values on a user-defined threshold (default 0.05), resulting in a set of candidate peaks with gaps between them (e). Small gaps are filled, resulting in a set of peak subgraphs (f). Graph Peak Caller finds a single “maximum path” (g, maximum path in blue) through each peak subgraph by selecting the path that has the highest number of input reads mapped to it. }
\label{fig1}
\end{figure}

The output from Graph Peak Caller consists of graph intervals, but the tool is also able to transform these into approximate positions on a linear reference genome (by projecting them to the nearest position on the linear reference genome), making it possible to analyse detected peaks further using existing “linear” approaches. Graph Peak Caller is also able to output candidates for differentially expressed peaks.

\begin{figure}[!h]
 \caption{{\bf Dna-binding motif enrichment plots.} The proportion of peaks enriched for a DNA-binding motif (Y-axis) when iteratively including more peaks from the total set of peaks sorted descending on score (X-axis) for MACS2 and Graph Peak Caller. }
\label{fig2}
\end{figure}


% Place tables after the first paragraph in which they are cited.
\begin{table}[!ht]
\begin{adjustwidth}{-2.25in}{0in} % Comment out/remove adjustwidth environment if table fits in text column.
\centering
\caption{My caption}
\label{my-label}
\begin{tabularx}{1.4\textwidth}{Xbccbccbb}
\toprule
& \multicolumn{3}{|l|}{Graph Peak Caller}  & \multicolumn{3}{l|}{MACS2} & \multicolumn{2}{l}{Bps. on linear ref.}                                                                                    \\ \midrule
\multicolumn{1}{l|}{Transcription factor} & Total & Shared & \multicolumn{1}{l|}{Unique} & Total & Shared & \multicolumn{1}{l|}{Unique} & Shared & Unique \\ \midrule


\multicolumn{1}{r|}{ERF115} & 24847 & 21164 (1846)&3683 (165) &23167 &21162 (1826) &2005 (132) &1.47 &3.62 \\
\multicolumn{1}{r|}{SEP3} &13902 &11304 (991) &2598 (123) &15517 &11302 (991) &4215 (161) &0.87 &2.59 \\
\multicolumn{1}{r|}{AP1} &15687 &12694 (1023)	 &2993 (160)	 &17030 &12692 (996)	 &4338 (159)	 &0.78	 &2.37 \\
\multicolumn{1}{r|}{SOC1} &14817	 &13854 (2258)	 &963 (136)	 &16407 &13849 (2242)	 &2558 (275)	 &1.04	 &2.35 \\
\multicolumn{1}{r|}{PI} &16548 &13526 (2171) &3022 (350)	 &16084 &13525 (2154)	 &2559 (218)	 &1.18	 &2.82 \\  \bottomrule
\end{tabularx}
\end{adjustwidth}
\end{table}


%PLOS does not support heading levels beyond the 3rd (no 4th level headings).
\subsection*{\lorem\ and \ipsum\ nunc blandit a tortor}
\subsubsection*{3rd level heading} 
Maecenas convallis mauris sit amet sem ultrices gravida. Etiam eget sapien nibh. Sed ac ipsum eget enim egestas ullamcorper nec euismod ligula. Curabitur fringilla pulvinar lectus consectetur pellentesque. Quisque augue sem, tincidunt sit amet feugiat eget, ullamcorper sed velit. Sed non aliquet felis. Lorem ipsum dolor sit amet, consectetur adipiscing elit. Mauris commodo justo ac dui pretium imperdiet. Sed suscipit iaculis mi at feugiat. 

\begin{enumerate}
	\item{react}
	\item{diffuse free particles}
	\item{increment time by dt and go to 1}
\end{enumerate}


\subsection*{Sed ac quam id nisi malesuada congue}

Nulla mi mi, venenatis sed ipsum varius, volutpat euismod diam. Proin rutrum vel massa non gravida. Quisque tempor sem et dignissim rutrum. Lorem ipsum dolor sit amet, consectetur adipiscing elit. Morbi at justo vitae nulla elementum commodo eu id massa. In vitae diam ac augue semper tincidunt eu ut eros. Fusce fringilla erat porttitor lectus cursus, vel sagittis arcu lobortis. Aliquam in enim semper, aliquam massa id, cursus neque. Praesent faucibus semper libero.

\begin{itemize}
	\item First bulleted item.
	\item Second bulleted item.
	\item Third bulleted item.
\end{itemize}

\section*{Discussion}
Nulla mi mi, venenatis sed ipsum varius, Table~\ref{table1} volutpat euismod diam. Proin rutrum vel massa non gravida. Quisque tempor sem et dignissim rutrum. Lorem ipsum dolor sit amet, consectetur adipiscing elit. Morbi at justo vitae nulla elementum commodo eu id massa. In vitae diam ac augue semper tincidunt eu ut eros. Fusce fringilla erat porttitor lectus cursus, vel sagittis arcu lobortis. Aliquam in enim semper, aliquam massa id, cursus neque. Praesent faucibus semper libero~\cite{bib3}.

\section*{Materials and methods}
\subsection*{Etiam eget sapien nibh}

% For figure citations, please use "Fig" instead of "Figure".
Nulla mi mi, Fig~\ref{fig1} venenatis sed ipsum varius, volutpat euismod diam. Proin rutrum vel massa non gravida. Quisque tempor sem et dignissim rutrum. Lorem ipsum dolor sit amet, consectetur adipiscing elit. Morbi at justo vitae nulla elementum commodo eu id massa. In vitae diam ac augue semper tincidunt eu ut eros. Fusce fringilla erat porttitor lectus cursus, \nameref{S1_Video} vel sagittis arcu lobortis. Aliquam in enim semper, aliquam massa id, cursus neque. Praesent faucibus semper libero.



\section*{Conclusion}

CO\textsubscript{2} Maecenas convallis mauris sit amet sem ultrices gravida. Etiam eget sapien nibh. Sed ac ipsum eget enim egestas ullamcorper nec euismod ligula. Curabitur fringilla pulvinar lectus consectetur pellentesque. Quisque augue sem, tincidunt sit amet feugiat eget, ullamcorper sed velit. 

Sed non aliquet felis. Lorem ipsum dolor sit amet, consectetur adipiscing elit. Mauris commodo justo ac dui pretium imperdiet. Sed suscipit iaculis mi at feugiat. Ut neque ipsum, luctus id lacus ut, laoreet scelerisque urna. Phasellus venenatis, tortor nec vestibulum mattis, massa tortor interdum felis, nec pellentesque metus tortor nec nisl. Ut ornare mauris tellus, vel dapibus arcu suscipit sed. Nam condimentum sem eget mollis euismod. Nullam dui urna, gravida venenatis dui et, tincidunt sodales ex. Nunc est dui, sodales sed mauris nec, auctor sagittis leo. Aliquam tincidunt, ex in facilisis elementum, libero lectus luctus est, non vulputate nisl augue at dolor. For more information, see \nameref{S1_Appendix}.

\section*{Supporting information}

% Include only the SI item label in the paragraph heading. Use the \nameref{label} command to cite SI items in the text.
\paragraph*{S1 Fig.}
\label{S1_Fig}
{\bf Bold the title sentence.} Add descriptive text after the title of the item (optional).

\paragraph*{S2 Fig.}
\label{S2_Fig}
{\bf Lorem ipsum.} Maecenas convallis mauris sit amet sem ultrices gravida. Etiam eget sapien nibh. Sed ac ipsum eget enim egestas ullamcorper nec euismod ligula. Curabitur fringilla pulvinar lectus consectetur pellentesque.

\paragraph*{S1 File.}
\label{S1_File}
{\bf Lorem ipsum.}  Maecenas convallis mauris sit amet sem ultrices gravida. Etiam eget sapien nibh. Sed ac ipsum eget enim egestas ullamcorper nec euismod ligula. Curabitur fringilla pulvinar lectus consectetur pellentesque.

\paragraph*{S1 Video.}
\label{S1_Video}
{\bf Lorem ipsum.}  Maecenas convallis mauris sit amet sem ultrices gravida. Etiam eget sapien nibh. Sed ac ipsum eget enim egestas ullamcorper nec euismod ligula. Curabitur fringilla pulvinar lectus consectetur pellentesque.

\paragraph*{S1 Appendix.}
\label{S1_Appendix}
{\bf Lorem ipsum.} Maecenas convallis mauris sit amet sem ultrices gravida. Etiam eget sapien nibh. Sed ac ipsum eget enim egestas ullamcorper nec euismod ligula. Curabitur fringilla pulvinar lectus consectetur pellentesque.

\paragraph*{S1 Table.}
\label{S1_Table}
{\bf Lorem ipsum.} Maecenas convallis mauris sit amet sem ultrices gravida. Etiam eget sapien nibh. Sed ac ipsum eget enim egestas ullamcorper nec euismod ligula. Curabitur fringilla pulvinar lectus consectetur pellentesque.

\section*{Acknowledgments}
Cras egestas velit mauris, eu mollis turpis pellentesque sit amet. Interdum et malesuada fames ac ante ipsum primis in faucibus. Nam id pretium nisi. Sed ac quam id nisi malesuada congue. Sed interdum aliquet augue, at pellentesque quam rhoncus vitae.

\nolinenumbers

% Either type in your references using
% \begin{thebibliography}{}
% \bibitem{}
% Text
% \end{thebibliography}
%
% or
%
% Compile your BiBTeX database using our plos2015.bst
% style file and paste the contents of your .bbl file
% here. See http://journals.plos.org/plosone/s/latex for 
% step-by-step instructions.
% 



\begin{thebibliography}{10}

\bibitem{rand}
Rand KD, Grytten I, Nederbragt AJ, Storvik GO, Glad IK, Sandve GK. 
\newblock {Coordinates and intervals in graph-based reference genomes}. 
\newblock BMC bioinformatics. 2017 Dec;18(1):263.

\bibitem{macs}
Zhang Y, Liu T, Meyer CA, Eeckhoute J, Johnson DS, Bernstein BE, Nusbaum C, Myers RM, Brown M, Li W, Liu XS. 
\newblock {Model-based analysis of ChIP-Seq (MACS)}. 
\newblock Genome biology. 2008 Nov;9(9):R137.

\bibitem{spp}
Kharchenko PV, Tolstorukov MY, Park PJ. Design and analysis of ChIP-seq experiments for DNA-binding proteins. Nature biotechnology. 2008 Dec;26(12):1351.

\bibitem{de_novo_assembly}
Iqbal Z, Caccamo M, Turner I, Flicek P, McVean G. De novo assembly and genotyping of variants using colored de Bruijn graphs. Nature genetics. 2012 Feb;44(2):226.

\bibitem{velvet}
Zerbino DR, Birney E. Velvet: algorithms for de novo short read assembly using de Bruijn graphs. Genome research. 2008 May 1;18(5):821-9.

\bibitem{vg}
Garrison E, Sirén J, Novak AM, Hickey G, Eizenga JM, Dawson ET, Jones W, Lin MF, Paten B, Durbin R. Sequence variation aware genome references and read mapping with the variation graph toolkit. bioRxiv. 2017 Jan 1:234856.

\bibitem{bwa_mem}
Li H. Aligning sequence reads, clone sequences and assembly contigs with BWA-MEM. arXiv preprint arXiv:1303.3997. 2013 Mar 16.

\bibitem{bowtie}
Langmead B, Salzberg SL. Fast gapped-read alignment with Bowtie 2. Nature methods. 2012 Apr;9(4):357.

\bibitem{fimo}
Grant CE, Bailey TL, Noble WS. FIMO: scanning for occurrences of a given motif. Bioinformatics. 2011 Feb 16;27(7):1017-8.

\bibitem{expresso}
Aghamirzaie D, Velmurugan KR, Wu S, Altarawy D, Heath LS, Grene R. Expresso: A database and web server for exploring the interaction of transcription factors and their target genes in Arabidopsis thaliana using ChIP-Seq peak data. F1000Research. 2017;6.

\bibitem{jaspar}
Sandelin A, Alkema W, {Engström} P, Wasserman WW, Lenhard B. 
JASPAR: an open\--access database for eukaryotic transcription factor binding profiles. Nucleic acids research. 2004 Jan 1;32:D91\--4.

\bibitem{trim_galore}
Krueger F. Trim galore. A wrapper tool around Cutadapt and FastQC to consistently apply quality and adapter trimming to FastQ files. 2015.

\bibitem{tair}
Arabidopsis Genome Initiative. Analysis of the genome sequence of the flowering plant Arabidopsis thaliana. nature. 2000 Dec;408(6814):796.

\bibitem{1000genomes}
1000 Genomes Project Consortium. A global reference for human genetic variation. Nature. 2015 Oct;526(7571):68.

\bibitem{genome_graphs}
Novak AM, Hickey G, Garrison E, Blum S, Connelly A, Dilthey A, Eizenga J, Elmohamed MS, Guthrie S, Kahles A, Keenan S. Genome graphs. bioRxiv. 2017 Jan 1:101378.

\end{thebibliography}



\end{document}

